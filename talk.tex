% \documentclass[12pt,notes,hyperref={unicode},aspectratio=169]{beamer}
\documentclass[12pt,presentation,hyperref={unicode},aspectratio=169]{beamer}
%\includeonlyframes{test}
\usepackage[utf8]{inputenc}

\usepackage[english]{babel}
\usepackage{forloop}
\usepackage{multirow}

\usepackage{amsmath,hyperref,multimedia, bbold, amssymb}

\usepackage{minted}
\usepackage{wrapfig, multicol}

% \AtBeginSection[]
% {
%   \begin{frame}<beamer>{Outline}
%     \tableofcontents[currentsection,currentsubsection]
%   \end{frame}
% }

\mode<presentation>
\usetheme{default}
\usecolortheme{beaver}
\setbeamertemplate{navigation symbols}{
  {\small
    \insertframenumber/\inserttotalframenumber
  }
}

\subtitle{Meeting C++ 2018}
\title{Competitive STL Extensions}
\author{Fedor Alekseev}
\institute{Moscow Institute of Physics and Technology: My pity}
\date{\today}

\newmintedfile[cppfile]{c++}{linenos}
\newmintedfile{asm}{linenos}

\begin{document}

\begin{frame}
  \titlepage
  \note{
    Good evening everyone!
    Thanks for having me.
    My name is Fedor.
    I'm a student from Moscow.

    I'm also doing some competitive programming as a hobby.
    I am a member of university team called My pity.

    This talk will be about some non-standard library tools that are available
    for g++ users and that are especially valuable for competitive programmers
  }
\end{frame}

% \begin{frame}{Outline}
%   \tableofcontents
% \end{frame}

\section{Competitive Programming}

\begin{frame}{Competitive Programming}
  \note{
    So, during a programming contest you are usually given a set of
    well-defined algorithmic problems.
    You or your team are usually given 4 to 12 problems and 2 to 5 hours to solve them.
    For each problem, you have to come up with a correct solution, and then
    code it and submit the source code to the judging system.
  }

  \begin{block}{A contest}
    \begin{itemize}
      \item<1-> Participants receive a set of 4--12 problems, and have just
        2--5 hours
      \item<2-> Problems are well-defined and usually have computer science
        nature
      \item<3-> Solving a problem means sending a program to the judging system.
        The program should pass all the secret test cases within time limit
      \item<4-> Optimal algorithmic complexity is usually enough, especially for
        C++ solutions
      \item<5-> Solutions are compiled in a judging environment without any
        additional libraries, with just the stock compiler installation
    \end{itemize}
  \end{block}
\end{frame}

% \begin{frame}{Competitive Programming}
%   Engineering is Programming integrated over time?
% \end{frame}

\section{Kool tricks}

\subsection{Standard library}
\begin{frame}[fragile]{Standard library}
  \begin{itemize}
    \item<1-> Algorithms: \mintinline{c++}|sort, lower_bound, unique, next_permutation,|~etc
    \item<1-> Data structures: \mintinline{bash}|{unordered_,}{set,map}|,
      simpler containers
    \item<2-> GNU C++ specific:
    % \begin{itemize}
      % \item<3->
      %   \mintinline{c++}|#include <algorithm>| includes
      %   \mintinline{c++}|std::__gcd|, even in pre-C++17 mode
      % \item<4->
        \mintinline{c++}|#include <bits/stdc++.h>| includes everything!
    % \end{itemize}
  \end{itemize}
\end{frame}

\subsection{g++ builtins}

\begin{frame}[fragile]{popcount: number of set bits}
  \cppfile{popcnt.cc}
  godbolts under x86 to
  \asmfile{popcnt.asm}
  Similarly,
  \mintinline{c++}|__builtin_clz| and \mintinline{c++}|__builtin_ctz|
  count leading/trailing zeros
\end{frame}

\subsection{SGI STL extensions}

% \begin{frame}[fragile]{SGI STL extensions: power}
%   \begin{block}{ext/numeric}
%     \cppfile[firstline=70,lastline=74,autogobble]{/usr/include/c++/8.2.1/ext/numeric}
%   \end{block}
% \end{frame}

\begin{frame}{SGI STL extensions: power}
  \begin{itemize}
    \item<1-> Sometimes you have an operation $(a^n)$ that can be
      expressed as some other operation $(a \cdot a)$ repeated $n$ times
    \item<1-> This is usually called exponentiation

    \item<2-> Matrix exponentiation:
      $A^n = E
        \cdot \underbrace{A \cdot A \cdot \ldots \cdot A}_{n \text{ times}}$,
        where $E$ is identity matrix
    \item<2-> Integer exponentiation over a modulus: $$
      a^n \mod{p} = 1
      \cdot \underbrace{(((a \mod{p}) \cdot a \mod{p}) \cdot
      \ldots \cdot a \mod{p})}_{n\text{ multiplications modulo }p}
    $$

    \item<3-> If multiplication is associative, this can be done in just
      $O(\log{n})$ multiplications
  \end{itemize}
\end{frame}

\begin{frame}[fragile]{SGI STL extensions: power}
  \cppfile[mathescape]{power_modulo.cc}
\end{frame}

% \begin{frame}{SGI STL extensions: rope}
%   rope?
% \end{frame}

\subsection{Policy-Based Data Structures}

\begin{frame}[fragile]{Policy-Based Data Structures}
  \begin{itemize}
    \item<1-> Policy-Based Data Structures library implements several types of
      search trees, hash tables, and heaps in an extensible way.
    \item<2-> Shipped with GNU C++ library as an
      extension within namespace
      \mintinline{c++}|__gnu_pbds|
  \end{itemize}
\end{frame}

\begin{frame}{PBDS: order statistics tree}
  \begin{itemize}
    \item<1->
      Usual std::set maintains a dynamic sorted sequence.
      It has methods like
      \mint{c++}|iterator set<T>::lower_bound(const T&) const|
      but no methods like
      \mint{c++}|iterator set<T>::at(size_t) const|

    \item<2->
      Efficient ($O(\log{n})$) implementation of these methods would require
      maintaining additional information in the search tree nodes

    \item<3->
      \mintinline{c++}|__gnu_pbds::tree_order_statistics_node_update|
      is a tree update policy that does exactly that, and enables
      methods
      \mint{c++}|tree::iterator tree::find_by_order(size_t) const|
      and
      \mint{c++}|size_t tree::order_of_key(const T&) const|
  \end{itemize}
\end{frame}

\begin{frame}[fragile]{PBDS: order statistics tree declaration}
  \cppfile[firstline=1,lastline=12]{order_statistics.cc}
\end{frame}

\begin{frame}[fragile]{PBDS: order statistics tree usage}
  \cppfile[firstline=15,lastline=27]{order_statistics.cc}
\end{frame}

\section{Lacking utilities}

\begin{frame}{Lacking utilities}
  \begin{itemize}
    \item<1-> C++ is a great language for competitive programming, but
    \item<2-> There are some lacking utilities that still constrain its dominance
    \item<3-> Most notably, arbitrary precision arithmetics:
      sometimes it is pragmatic to switch to Python or Java just for big
      integers
  \end{itemize}
\end{frame}

\begin{frame}{kthxbye}
  \begin{itemize}
    \item Thanks!
    \item More examples:
      \url{https://github.com/moskupols/competitive-stl-extensions}
    \item For more info on PBDS see GNU C++ library manual:
      \url{https://goo.gl/PmR86Z}
  \end{itemize}
\end{frame}

\end{document}
